\documentclass[12pt]{article}
\usepackage[utf8]{inputenc}
\usepackage{geometry}
\geometry{a4paper, centering, scale=0.8}

\title{Embedded System Design Workshop 1\\
Pre-workshop Question \\
Wednesday 12:00}

\author{
    Hongbo Zhou 1067814
}

\date{}

\begin{document}
\maketitle 

\section*{Question 1}
Statecharts is actually a kind of "hierarchical" state machine that can obtain nested state machine, called "substate". 
StateChart can be in multiple states at any given time while FSM can only be in exactly one state.

\section*{Question 2}
A region is a container for one or more states, a state is the condition that the statechart is in. States in one region are parallel.

\section*{Question 3}
An arrow containing three open or closed cells represents one transition between states. \\
The left rectangle indicates the presence of a trigger.\\
The middle rectangle indicates the presence of a guard.\\
The right rectangle indicates the presence of an action.\\
The arrow connects two states.


\end{document}